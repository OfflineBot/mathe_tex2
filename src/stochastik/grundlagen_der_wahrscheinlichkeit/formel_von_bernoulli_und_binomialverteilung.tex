\subsection{Formel von Bernoulli und Binomialverteilung}
\label{sec:bernoulli}
\textbf{Anwendungen}:

Die Formel von Bernoulli wird bei Wahrscheinlichkeiten wie einem Münzwurf eingesetzt. Das heißt bei Wahrscheinlichkeiten die immer die selbe Chance haben (Münzwurf $50\%$ pro Seite)

\vspace{0.4cm}

\textbf{Formel}:
\[
    P(X = k) = 
    \begin{pmatrix}
        n \\ k
    \end{pmatrix}
    \cdot p^k \cdot q^{n-k}
\]

Hierbei bedeuten:
\begin{itemize}
    \item $
    \begin{pmatrix}
        n \\ k
    \end{pmatrix}
    $
    (ausgesprochen "n über k"). 
    $n$ ist anzahl der Versuche und $k$ die anzahl der Möglichkeiten. Wird wie folgt berechnet: 
    \[
        \begin{pmatrix}
            n \\ k
        \end{pmatrix}
        =
        \frac{n!}{k!(n-k!)}
    \]

    \item $p^k$: Die Wahrscheinlichkeit, dass $k$ Erfolge auftreten.
    \item $q^{n-k}$: Auch $(1-p)^{n-k}$ geschrieben. Gibt die Misserfolge an. (Restwahrscheinlichkeit um auf $100\%$ zu kommen)
\end{itemize}

Also ist $P(X=k)$ die Wahrscheinlichkeit, dass in $n$ Versuchen genau $k$ Erfolge erzielt werden, wenn die Erfolgswahrscheinlichkeit in jedem Versuch $p$ beträgt.

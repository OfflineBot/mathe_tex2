\subsection{Flächeninhalt}
Um den Flächeninhalt eines Graphen zu bestimmen berechnet man die Stammfunktion des Graphen. 
Dann berechnet man das Integral von $a$ zu $b$ um die Fläche zu bestimmen. 

\underline{Beispiel}: \\
Funktion: 
$
f(x) = x^2
$
\\
Gesucht ist der Flächeninhalt von 1 bis 4. Somit: 

$\int_1^4 x^2\ dx $

$\rightarrow [\frac{1}{3}x^3]_1^4$ $|$ Aufgeleitet

$\rightarrow (\frac{1}{3}4^3) - (\frac{1}{3}1^3) = 21$ $|$ Eingesetzt

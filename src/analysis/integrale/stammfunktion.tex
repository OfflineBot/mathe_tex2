\subsection{Stammfunktion}
Auch als Aufleiten bekannt.
\subsubsection{(Aufleitungs-) Regeln}
\label{sec:aufleitungs_regeln}
Aufbau: Name der Regel. \\
Erste Formel die allgemeine Formel (neutral). \\
Zweite Formel (optional) ein Zahlenbeispiel. 
\par
\underline{Konstantenregel}: \\
$f(x) = n\ \rightarrow\ F(x) = n\cdot x + C$ \\
Dabei ist $n$ eine konstante Zahl. \\
$f(x) = 3\ \rightarrow\ F(x) = 3x + C$ 
\par
\underline{Faktorregel}: \\
$f(x) = a\cdot x \rightarrow F(x) = a\cdot X + C$ \\
Dabei ist $x$ ein beliebiger Wert wie: $x^2, 3x, etc.$ und $X$ ist die Aufleitung des Wertes. 
\par
\underline{Potenzregeln}: \\
$f(x) = x^n\ \rightarrow\ F(x) = \frac{1}{n + 1}\cdot x^{n + 1} + C = \frac{x^{n+1}}{n+1} + C$\\
Dabei ist $n$ eine konstante Zahl. \\
$f(x) = a\cdot x^3\ \rightarrow\ F(x) = a\cdot \frac{1}{3 + 1}\cdot x^{3 + 1} + C= a\cdot \frac{x^{4}}{4} + C$ 
\par
\underline{Summenregel}: \\
$f(x) = h + k \rightarrow F(x) = H + K$ \\
Dabei ist $h$ und $k$ ein beliebiger Wert wie z.B.: $x^2$, $3x$, $4x^34$, etc. 
\par
\underline{Zu Beachten}:
\begin{itemize}
    \item Aufleitung von $e$. ($e^x$ bleib unverändert)
    \item Wuzeln können auch als Exponent geschrieben werden ($^n\sqrt{x} = x^{\frac{1}{n}}$)
    \item $ln(x)$ aufgeleitet ist $e^x$
\end{itemize}

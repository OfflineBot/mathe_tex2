\subsection{Stammfunktion}
\subsubsection{Regeln}
\label{sec:aufleitungs_regeln}
Konstanten aufleiten: \\
$f(x) = n\ \rightarrow\ F(x) = n\cdot x + C$ \\
Dabei ist $n$ eine konstante Zahl. \\
$f(x) = 3\ \rightarrow\ F(x) = 3x + C$ \\\\
Faktorregel: \\
$f(x) = a\cdot x \rightarrow F(x) = a\cdot X + C$ \\
Dabei ist $x$ ein beliebiger Wert (siehe Summenregel für Beispiele) und $X$ ist die Aufleitung des Wertes. \\\\
Potenzregeln: \\
$f(x) = x^n\ \rightarrow\ F(x) = \frac{1}{n + 1}\cdot x^{n + 1} + C = \frac{x^{n+1}}{n+1} + C$\\
Dabei ist $n$ eine konstante Zahl. \\
$f(x) = a\cdot x^3\ \rightarrow\ F(x) = a\cdot \frac{1}{3 + 1}\cdot x^{3 + 1} + C= a\cdot \frac{x^{4}}{4} + C$ \\\\
Summenregel: \\
$f(x) = h + k \rightarrow F(x) = H + K$ \\
Dabei ist $h$ und $k$ ein beliebiger Wert wie z.B.: $x^2$, $3x$, $4x^34$, etc. \\

\subsubsection{Zu Beachten}
Aufleitung der Zahl: $e$. \\
Wie bei Ableitung bleibt $e^x$ unverändert
\\\\
Aufleitung von Wurzeln.\\
$\rightarrow\ \sqrt{x} \rightarrow\ x^{\frac{1}{2}}$, Oder: $^n\sqrt{x} \rightarrow\ x^{\frac{1}{n}}$

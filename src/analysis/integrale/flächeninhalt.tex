\subsection{Flächeninhalt}
Um den Flächeninhalt eines Graphen zu bestimmen berechnet man die Stammfunktion des Graphen. 
Dann berechnet man das Integral von $a$ zu $b$ um die Fläche zu bestimmen. 

\textbf{Beispiel}: \\
Funktion: 
$f(x) = x^2$ \\
Gesucht ist der Flächeninhalt von 1 bis 4. Somit: 
\[
\begin{array}{l l}
f(x) = \int_1^4 x^2 \, dx & \\
\implies \left[\frac{1}{3}x^3\right]_1^4 & |\ \text{ Aufgeleitet} \\
\implies \left(\frac{1}{3}4^3\right) - \left(\frac{1}{3}1^3\right) & |\ \text{ Eingesetzt}
\end{array}
\]


\subsection{Graphen}

\parbox[t]{0.3\textwidth}{
    \underline{Exponentielle Steigung}: 

    \scalebox{0.6}{
        \begin{tikzpicture}
            \begin{axis}[
                axis lines=middle,
                xlabel={$x$},
                ylabel={$y$},
                title={Funktion: $f(x) = e^x$},
                %grid=major,
                domain=-2:2, % Range for x
                samples=100, % Number of samples for smoothness
                ]
                \addplot[blue, thick] {exp(x)};
                %\legend{$y = e^x$}
            \end{axis}
        \end{tikzpicture}
    }
}
\hfill
\parbox[t]{0.3\textwidth}{
    \underline{Exponentieller Zerfall}: 

    \scalebox{0.6}{
        \begin{tikzpicture}
            \begin{axis}[
                axis lines=middle,
                xlabel={$x$},
                ylabel={$y$},
                title={Funktion: $f(x) = e^{-x}$},
                %grid=major,
                domain=-2:2, % Range for x
                samples=100, % Number of samples for smoothness
                ]
                \addplot[blue, thick] {exp(-x)};
                %\legend{$y = e^x$}
            \end{axis}
        \end{tikzpicture}
    }
}
\hfill
\parbox[t]{0.3\textwidth}{
    \underline{Konstant (nicht Exponential)}:

    \scalebox{0.6}{
        \begin{tikzpicture}
            \begin{axis}[
                axis lines=middle,
                xlabel={$x$},
                ylabel={$y$},
                title={Funktion: $f(x) = e^{1}$},
                %grid=major,
                domain=-2:2, % Range for x
                samples=100, % Number of samples for smoothness
                ]
                \addplot[blue, thick] {exp(1)};
                %\legend{$y = e^x$}
            \end{axis}
        \end{tikzpicture}
    }
}


\subsection{Umkehrfunktion}
Umkehrfunktion ist die Umkehrung einer Funktion. \\
Dabei wird ein gegebener Term wie $f(x) = x^2$ genommen und $f(x)$ mit $y$ und und das $x^2$ durch $y^2$ ersetzt. \\
\[
\begin{gathered}
    f(x) = y^2 \\
    \rightarrow y = x^2
\end{gathered}
\]
Dann wird nach $x$ aufgelößt.

Beispiel anhand von Zahlenbeispiel:
\begin{enumerate}
    \item Gegeben Funktion: $f(x) = 3x^2$
    \item Umbenennung: $y = 3x^2$
    \item Nach $x$ Auflösen: $3x^2 = y$
    \item Ausgleichen: $x = \sqrt{\frac{y}{3}}$
    \item Ergebnis: $f(x) = \sqrt{\frac{x}{3}}$
\end{enumerate}

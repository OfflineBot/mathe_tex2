\subsection{Extrem- und Wendepunkte}
\underline{Extrempunkte}: \\
Extrempunkte sind Punkte auf dem Graphen einer Funktion, 
an denen die Funktion entweder ein lokales Maximum oder Minimum erreicht.
\begin{itemize}
    \item \underline{Lokales Maximum}: 
        An einem lokalen Maximum ist der Funktionswert größer als in der unmittelbaren Umgebung. Mathematisch bedeutet dies, 
        dass die erste Ableitung der Funktion an diesem Punkt null ist ($f'(x)=0$) und die zweite Ableitung negativ ist ($f''(x)<0$).
    \item \underline{Lokales Minimum}: 
        An einem lokalen Minimum ist der Funktionswert kleiner als in der unmittelbaren Umgebung. 
        Hier ist ebenfalls die erste Ableitung null ($f'(x)=0$), 
        jedoch ist die zweite Ableitung positiv ($f''(x)>0$).
\end{itemize} 
\
\\\\
\underline{Wendepunkte}: \\
Ein Wendepunkt ist ein Punkt auf dem Graphen einer Funktion, 
an dem sich das Krümmungsverhalten ändert. 
Das bedeutet, die Funktion wechselt an diesem Punkt von konkav zu konvex oder umgekehrt.
\begin{itemize}
    \item \underline{Bestimmung von Wendepunkten}: 
        Ein Wendepunkt liegt vor, 
        wenn die zweite Ableitung der Funktion null ist ($f''(x)=0$) und die dritte Ableitung nicht null ist ($f'''(x)\neq 0$). 
\end{itemize}

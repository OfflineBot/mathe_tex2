\subsection{Extremwertprobleme mit Nebenbedingungen}
Extremwertprobleme mit Nebenbedingungen sind mathematische Probleme, 
bei denen eine Zielfunktion unter Berücksichtigung zusätzlicher Bedingungen optimiert werden soll. 
Diese werden häufig in Form von Textaufgaben verwendet oder einfach mit einer Formel und mithilfe von Text.
\par
\secspacebig
\underline{\textbf{!!! Offen für neue + bessere Beispiele !!!}} 
\par
\textbf{Beispiel}: \\
Ein Unternehmen produziert zwei Arten von Produkten, X und Y. 
Der Gewinn pro verkaufter Einheit beträgt 10 Euro für Produkt X und 15 Euro für Produkt Y. 
Das Unternehmen möchte die Gesamtgewinnmarge maximieren und muss gleichzeitig sicherstellen, 
dass mindestens 100 Einheiten des Produkts X verkauft werden.
\begin{itemize}
    \item \textbf{Zielfunktion:} Maximiere die Gesamtgewinnmarge $G = 10x + 15y$, 
        wobei $x$ die Anzahl der Einheiten von Produkt X und $y$ die Anzahl der Einheiten von Produkt Y ist.
    \item \textbf{Nebenbedingung:} Verkaufsmenge von X muss mindestens 100 Einheiten betragen $x \geq 100$
\end{itemize}
Lösungsschritte:
\begin{enumerate}
    \item Formulieren der Zielfunktion und Nebenbedingung: \\
        $G = 10x + 15y$
    \item Lösen: \\
        Wegen Nebenbedingung setze $x = 100$ somit: \\
        $G = 10 \cdot 100 + 15y = 1000 + 15y$
\end{enumerate}
Also ist das Ergebnis: $G = 1000 + 15y$ \\\\
\

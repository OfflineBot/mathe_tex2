\subsection{Tangente und Normale}
\label{sec:tangenteundnormale}
\textbf{Tangente}: \\
Eine Tangente an den Graphen einer Funktion an einem bestimmten Punkt ist eine Gerade, 
die den Graphen genau an diesem Punkt berührt. 
Die Steigung der Tangente entspricht der Ableitung der Funktion an diesem Punkt. \\
Um die Tangente bestimmen zu können ($f(x) = m\cdot x + c$):
\begin{itemize}
    \item $m$: $f'(\alpha)$. Bei dem $\alpha$ der x-Wert ist, an dem die Tangente gesucht wird.
    \item $c$: $f(\alpha)$. Bei dem $\alpha$ der x-Wert ist, an dem die Tangente gesucht wird.
\end{itemize} 
\
\\
\textbf{Normale}: \\
Eine Normale ist ein Graph der Orthogonal zur Tangente verläuft. 
Auch dieser wird durch die Tangentengleichung ($f(x) = m\cdot x + c$) bestimmt und kann von der Tangente bestimmt werden. \\
Tangente: $f(x) = m \cdot x + c$ \\
Normale: 
$
f(x) = -\frac{1}{m} \cdot x + c
$

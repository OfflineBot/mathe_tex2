\subsection{Ableitungsregeln}
\label{sec:ableitungs_regeln}
Aufbau: Name der Regel. \\
Erste Formel die allgemeine Formel (neutral). \\
Zweite Formel (optional) ein Zahlenbeispiel.
\par
\underline{Konstantenregel}: \\
$f(x) = n \rightarrow f'(x) = 0$ \\
$f(x) = 3 \rightarrow f'(x) = 0$ 
\par
\underline{Potenzregel}: \\
$f(x) = x^n \rightarrow f'(x) = n\cdot x^{n - 1}$ \\
$f(x) = x^5 \rightarrow f'(x) = 5x^{4}$ 
\par
\underline{Faktorregel}: \\
$f(x) = a\cdot g(x) \rightarrow f'(x) = a\cdot g'(x)$ \\
$f(x) = 3\cdot x^2 \rightarrow f'(x) = 3\cdot 2\cdot x^1 = 6x$ 
\par
\underline{Summenregel/Differenzregel}: \\
$f(x) = g(x) + h(x) \rightarrow f'(x) = g'(x) + h'(x)$ \\
$f(x) = g(x) - h(x) \rightarrow f'(x) = g'(x) - h'(x)$ 
\par
\underline{Produktregel}: \\
$f(x) = u(x) \cdot v(x) \rightarrow f'(x) = u'(x) \cdot v(x) + u(x) \cdot v'(x)$ 
\par
\underline{Kettenregel}: \\
$f(x) = u(v(x)) \rightarrow f'(x) = u'(v(x)) \cdot v'(x)$

\subsubsection{Zu Beachten}
\begin{itemize}
    \item $ln(x)$ abgeleitet ist $\frac{1}{x}$
    \item $e^x$ abgeleitet bleibt gleich ($e^x$)
\end{itemize}


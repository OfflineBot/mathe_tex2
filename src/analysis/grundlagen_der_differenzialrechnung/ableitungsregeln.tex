\subsection{Ableitungsregeln}
\label{sec:ableitungs_regeln}
Aufbau: Name der Regel. \\
Erste Formel die allgemeine Formel (neutral). \\
Zweite Formel (optional) ein Zahlenbeispiel.

\begin{multicols}{2}

\textbf{Konstantenregel}: \\
$
f(x) = n \\
\implies f'(x) = 0
$

\vspace{0.5cm}

\textbf{Potenzregel}: \\
$
f(x) = x^n \\
\implies f'(x) = n\cdot x^{n - 1}
$

\vspace{0.5cm}

\textbf{Faktorregel}: \\
$
f(x) = a\cdot g(x) \\
\implies f'(x) = a\cdot g'(x)
$

\columnbreak

\textbf{Summenregel/Differenzregel}: \\
$
f(x) = g(x) + h(x) \\
\implies f'(x) = g'(x) + h'(x) 
$

\vspace{0.5cm}

\textbf{Produktregel}: \\
$
f(x) = u(x) \cdot v(x) \\
\implies f'(x) = u'(x) \cdot v(x) + u(x) \cdot v'(x)
$ 

\vspace{0.5cm}

\textbf{Kettenregel}: \\
$
f(x) = u(v(x)) \\
\implies f'(x) = u'(v(x)) \cdot v'(x)
$

\end{multicols}

\subsubsection{Zu Beachten}
\begin{itemize}
    \item $ln(x)$ abgeleitet ist $\frac{1}{x}$
    \item $e^x$ abgeleitet bleibt gleich ($e^x$). Bei $e$ wird die Kettenregel angewendet.
\end{itemize}


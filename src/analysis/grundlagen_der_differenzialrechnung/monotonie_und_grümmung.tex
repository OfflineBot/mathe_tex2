\subsection{Monotonie und Grümmung}
\underline{Montonie}: \\
Monotonie bezieht sich auf das Verhalten einer Funktion in Bezug darauf, 
ob sie stetig zunimmt oder abnimmt. 
Eine Funktion wird als monoton steigend bezeichnet, 
wenn für zwei Punkte $x_1$ und $x_2$ mit $x_1 < x_2$ der Funktionswert an $x_1$ kleiner oder gleich dem Funktionswert an $x_2$ ist. 
Umgekehrt wird eine Funktion als monoton fallend bezeichnet, 
wenn für zwei Punkte $x_1$ und $x_2$ mit $x_1>x_2$ der Funktionswert an $x_1$ größer oder gleich dem Funktionswert an $x_2$ ist. \\\\
\
\underline{Krümmung}: \\
Die Krümmung einer Funktion beschreibt, wie stark eine Kurve von einer Geraden abweicht. 
Mathematisch gesehen wird die Krümmung einer Funktion durch die zweite Ableitung der Funktion beschrieben. 
Eine positive zweite Ableitung bedeutet, 
dass die Funktion eine nach oben geöffnete Krümmung (konkav) hat, 
während eine negative zweite Ableitung eine nach unten geöffnete Krümmung (konvex) anzeigt. 
Eine Krümmung von null bedeutet, dass die Funktion an dieser Stelle eine Wendepunkt hat, 
wo die Krümmung ihre Richtung ändert.

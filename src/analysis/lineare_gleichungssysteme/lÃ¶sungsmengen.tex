\subsection{Lösungsmenge linearer Gleichungssysteme + Bestimmtheit}

\underline{Eindeutige Lösung:} \\
Es gibt genau einen Satz von Werten, der alle Gleichungen erfüllt. \\
Ein zwei Dimensionen würde es bedeuten, dass sich die Linien in einem Punkt schneiden würden. \\
(\textbf{genau einen Schnittpunkt})

Beispiel: 

$
x + y = 5 \\
x - y = 1 
$

Man kann diese Gleichung nur mit $x = 3$ und $y = 2$ Lösen. \\
Deshalb eindeutig bestimmbar.

\bigskip

\underline{Unendlich viele Lösungen:} \\
Es gibt verschiedene Sätze von Werten, die die Gleichungen erfüllen. \\
Dies passiert, wenn die Linien identisch sind. \\
(\textbf{unendlich viele Schnittpunkte})

Beispiel:

$
2x + 4y = 8 \\
x + 2y = 4
$

Die zwei Gleichungen sind ein Vielfaches von einander. \\
Somit sind diese identisch. 

\bigskip

\underline{Keine Lösung:} \\
Es gibt keinen einzigen Satz von Werten, der alle Gleichungen erfüllt. \\
Das passiert, wenn die Linien parallel sind. \\
(\textbf{keinen Schnittpunkt})

Beispiel:

$x + y = 2 \\ 
x + y = 5$

Diese Gleichung ist nicht lösbar. \\
Somit hat diese auch keine Lösung.

\bigskip

\underline{Unterbestimmt:} \\
Es gibt nur 2 valide Gleichungen in einem 3 Dimensionalen Gleichungssystem. \\
Wenn eine Gleichung ein Vielfaches einer anderen ist.

Beispiel:

$
x + y + 4z = 5 \\
2x + 2y + 8z = 10\ |\ $Vielfaches von der ersten Gleichung$ \\
3x + 1y - z = 3
$

Das System hat nicht genügend Information um eindeutig bestimmt zu werden. \\
Es gibt unendlich viele Lösungen. \\
Man kann diese Gleichung lösen indem man Parameter einsetzt. 
Mehr dazu \hyperref[sec:lineare_parameter]{hier}.

\bigskip

\underline{Überbestimmt:} \\
Es gibt 3 Gleichungen in einem 2 Dimensionalen Gleichungssystem.

Beispiel: 

$
x + y = 4 \\
3y - 2y = 9 \\
-2x + 2y = 1
$

Drei unterschiedliche Gleichungen jedoch in einem 2 dimensionalen Gleichungssystem und somit überbestimmt. \\
Es ist eine geringe Wahrscheinlichkeit einer Lösung.


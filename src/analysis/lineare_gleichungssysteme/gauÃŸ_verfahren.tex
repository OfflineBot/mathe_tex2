\subsection{Gauß-Verfahren}
Das Gauß-Verfahren ist eine Art der Lösung eines linearen Gleichungssystem. 
Das Gauß-Verfahren wird mithilfe der Matrixschreibweise ausgerechnet.

Hier ist nur eine ganz kurze Erklärung! 
Für mehr \href{https://simpleclub.com/lessons/mathematik-gaussalgorithmus}{hier} Klicken.

\underline{Beispiel}: \\
Das Gleichungssystem: 

\begin{align*}
    2x + 3y - z &= 5 \\
    4x - y + 5z &= 6 \\
    -2x + 2y + 3z &= -4
\end{align*}

Vereinfachen zur Matrix: 

$$
\left[
\begin{array}{ccc|c}
    2 & 3 & 1 & 5 \\
    4 & -1 & 5 & 6 \\
    -2 & 2 & 3 & -4
\end{array}
\right]
$$

Dabei ist es wichtig \textbf{Nullen} zu erschaffen indem man Zeilen miteinander \textbf{Addiert} oder \textbf{Subtrahiert}. \\
Man kann auch einzelne Zeilen \textbf{Multiplizieren} oder \textbf{Dividieren} um dies zu vereinfachen/möglich machen. \\
Die Nullen müssen ein einer Pyramidenartigen anordnung bestehen um das Ergebnis (rechte Seite) eindeutig bestimmen zu können. 

Etwa so: \\
\textbf{Wichtig:} Zahlen werden hier nur ersetzt und nicht wirklich ausgerechnet!

$$
\left[
\begin{array}{ccc|c}
    2 & 3 & 1 & 5 \\
    0 & -1 & 5 & 6 \\
    0 & 0 & 3 & -4
\end{array}
\right]
$$

Damit kann man die Gleichung: $3x_3 = -4$ aufstellen um $x_3$ zu bekommen. \\
Danach hat man $-1x_2 + 5x_3 = 6$ und kann für $x_3$ das Ergebnis von davor einsetzen. 

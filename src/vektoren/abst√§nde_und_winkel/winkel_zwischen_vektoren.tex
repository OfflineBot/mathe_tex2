\subsection{Winkel zwischen Vektoren}
\label{sec:winkel_vektoren}
Auch Schnittwinkel genannt. Mithilfe der Formel des Skalarprodukts herleitbar. 

\[\vec{a} \bullet \vec{b} = |\vec{a}| \cdot |\vec{b}| \cdot cos(\alpha) \]
\[\rightarrow cos(\alpha) = \frac{\vec{a}\bullet \vec{b}}{|\vec{a}|\cdot |\vec{b}|} \]
Dabei steht $cos(\alpha)$ für den Winkel und kann mithilfe des Taschenrechners mit $cos^{-1}(\sigma)$ in Grad umgewandelt werden. 
$\sigma$ steht hierbei für $cos(\alpha)$ 

Zahlenbeispiel: 
\[
\vec{a} = 
\begin{Bmatrix}
    5 \\ 0 \\ 1
\end{Bmatrix} 
\ \ \ \
\vec{b} = 
\begin{Bmatrix}
    2 \\ 3 \\ 2
\end{Bmatrix} \\
\]
Berechne benötigte Werte:
\begin{itemize}
    \item $\vec{a} \bullet \vec{b}$ = $5 \cdot 2 + 0 \cdot 3 + 1 \cdot 2 = 12$
    \item $|\vec{a}|$ = $\sqrt{5^2 + 0^2 + 1^2} = \sqrt{11}$
    \item $|\vec{b}|$ = $\sqrt{2^2 + 3^2 + 2^2} = \sqrt{17}$
\end{itemize}
Setze ein: 

\[cos(\alpha) = \frac{12}{\sqrt{11}\cdot \sqrt{17}} = 0.877 \]
Gebe in Taschenrechner ein: 
\[cos^{-1}(0.877) = 28.717 \]
Somit ist der Winkel: $28.717$°

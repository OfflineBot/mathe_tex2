\subsection{Abstand von Punkt zu Gerade}
\underline{Erklärung}: \\
Um den Abstand von einem Punkt zu einer Geraden zu berechnen erstellt man eine Hilfsebene.
Diese dient dazu, den Durchstoßpunkt und somit den Lotfußpunkt zu finden (kürzester Weg zur Geraden). 
Die Ebene wird erst durch die Normalenform und damit in die Koordinatenform umgeformt. \\
Ca so: \\
$
(\vec{x} - \vec{p}) \bullet \vec{n} \\
\rightarrow \vec{x} \bullet \vec{n} = \vec{p} \bullet \vec{n} \\
\rightarrow x_1a + x_2b + x_3c = d
$
\\
Dabei steht: 
\begin{itemize}
    \item $\vec{x}$: Einzusetzender Punkt (wird später die Gerade in Koordinatenform eingesetzt)
    \item $\vec{n}$: Normalenvektor. Ist der Richtungsvektor von Geraden 
    \item $\vec{p}$: Stützvektor. Ist der gegebene Punkt
\end{itemize}
Damit kann man dann die komplette Gerade $\vec{x}$ einsetzen 
(jeweils für die Axen) und letztendlich ausgleichen. 
Das Ergebnis (eine Zahl) wird dann für $\lambda$ eingesetz und hat somit $F$.
Mit dem Ergebnis: $F$ und dem Punkt $P$ berechnen man den Betrag $|\overrightarrow{PF}|$. \\
Damit hat man den Abstand von Punkt zu Gerade.
\\\\
\underline{Zahlenbeispiel}: \\
Gegeben: 
\begin{itemize}
    \item Punkt: $P(5|6|-1)$
    \item Gerade: $
        \vec{g} = 
        \begin{Bmatrix}
            1 \\ 1 \\ 1
        \end{Bmatrix}
        + \lambda \cdot
        \begin{Bmatrix}
            6 \\ 3 \\ 6
        \end{Bmatrix}
        $
\end{itemize}
Durchführung: \\
Aufstellen der Normalenform: \\
$
E: (\vec{x} - \vec{p}) \bullet \vec{n} = 0\\
E: 
\begin{pmatrix}
    \begin{Bmatrix}
        x_1 \\ x_2 \\ x_3
    \end{Bmatrix}
    - 
    \begin{Bmatrix}
        1 \\ 1 \\ 1
    \end{Bmatrix}
\end{pmatrix}
\bullet
\begin{Bmatrix}
    6 \\ 3 \\ 6
\end{Bmatrix}
= 0 \\
\Rightarrow
\vec{x} \bullet \vec{n} = \vec{p} \bullet \vec{n} \\
= x_1\cdot 6 + x_2 \cdot 3 + x_3 \cdot 6 = 42
$
\\
\
\\
Setze Gerade in $\vec{x}$ ein: \\
$
6(1+6s) + 3(1+3s) + 6(1+6s) = 42 \\
6 + 3 + 6 + 6s + 3s + 6s = 42 \\
15 + 81s = 42
s = \frac{1}{3}
$
\\
\
\\ 
Setzte das Ergebnis für $\lambda$ ein und erhalte $F$: \\
$
g(x) = 
\begin{Bmatrix}
    1 \\ 1 \\ 1
\end{Bmatrix}
+ \frac{1}{3} \cdot
\begin{Bmatrix}
    6 \\ 3 \\ 6
\end{Bmatrix}
= 
\begin{Bmatrix}
    3 \\ 2 \\ 3
\end{Bmatrix}
$
\\
\
\\
Berechne $|\overrightarrow{PF}|$: \\
$\sqrt{36} = 6$
\\
Somit ist die Länge: $6$



\subsection{Spiegelung und Symmetrie}
\textbf{Spielgelungsmöglichkeiten}:
    
\begin{itemize}
    \item Über eine Achse: \\
        Beim spiegeln von einer Achse kehrt man die anderen Achsen um.

        \textbf{Beispiel}: 
        \begin{itemize}
            \item Spiegelung auf $x_1$-Ebene für Punkt $P(1|2|3) \implies P'(1|-2|-3)$ 
            \item Spiegelung auf $x_2$- und $x_3$-Ebene für Punkt $P(1|2|3) \implies P'(-1|2|3)$
        \end{itemize}

    \item Über den Ursprung ($(0|0|0)$): \\
        Beim spiegeln auf dem Ursprung werden alle Werte umgekehrt.

        \textbf{Beispiel}:
        \begin{itemize}
            \item Spiegeln von Punkt $P(1|2|3) \implies P'(-1|-2|-3)$
        \end{itemize}

    \item Über einen anderen Punkt: \\
        Man berechnet den Richtungsvektor $\vec{R}$ von Punkt $P$ zu dem Spiegelpunkt $Z$ und berechnet $P+2\vec{R}$

    \textbf{Beispiel}:
    \begin{itemize}
        \item Spiegelung von Punkt $P(1|2|3)$ über Spiegelpunkt $Z(4|5|4)$ \\
            $\implies \vec{R} = \overrightarrow{PZ} = 
            \begin{pmatrix}
                4 - 1 \\ 5 - 2 \\ 4 - 3
            \end{pmatrix}
            =
            \begin{pmatrix}
                3 \\ 3 \\ 1
            \end{pmatrix}
            $
            
        Berechne den Spiegelpunkt $P+2\vec{R}$:

        $P'(1 + 2\cdot 3|2 + 2\cdot 3| 3 + 2\cdot 1) = P'(7|8|5)$

    \end{itemize}
\end{itemize}


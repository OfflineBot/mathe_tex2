\subsection{Hess'sche Normalenform: Abstand von Punkt zu Ebene}
\label{sec:hessscheform}

\textbf{Von der Normalenform:} \\
Um den Abstand von einem Punkt zur gerade zu berechnen ersetzt man $\vec{x}$ durch den Punkt und teilt den Normalenvektor durch den Betrag des  Normalenvektor (Normalisieren). Am Ende muss man noch den Betrag von dem Ergebnis (ein Vektor) gezogen werden und man hat die Länge.

\[
\begin{aligned}
    &((\vec{x} - \vec{p}) \bullet \vec{n}) \newline \\
    \implies &((\vec{r} - \vec{p}) \bullet \vec{n_0}) = b \\
    \implies &|((\vec{r} - \vec{p}) \bullet \vec{n_0})| = d
\end{aligned}
\]

Dabei ist $b$ der Abstand, $d$ wäre die Länge und $\vec{n_0} = \frac{\vec{n}}{|\vec{n}|}$.

\vspace{0.5cm}

\textbf{Von der Koordinatenform:} \\
Um den Abstand von einem Punkt zur Ebene in der Koordinatenform zu berechenen, formt man die Koordinatenform um und setzt für $\vec{x}$ den Punkt ($\vec{r}$) ein.

\[
\begin{aligned}
    &n_1\cdot r_1 + n_2\cdot r_2 + n_3\cdot r_3 = b \\
    \implies &\frac{\vec{n}\cdot \vec{r} - b}{\vec{n_0}} = b \\
    \implies &|\frac{\vec{n}\cdot \vec{r} - b}{\vec{n_0}}| = d
\end{aligned}
\]

Dabei ist $b$ der Abstand, $d$ wäre die Länge und $\vec{n_0} = \frac{\vec{n}}{|\vec{n}|}$ 

\vspace{0.5cm}

\textbf{Mit Lotfußpunkt:} \\
Man nimmt den Betrag von dem Lotfußpunkt $\vec{L}$ und dem Punkt $\vec{P}$. Dann hat man: 

\[
\begin{aligned}
    |\overrightarrow{PL}|
\end{aligned}
\]


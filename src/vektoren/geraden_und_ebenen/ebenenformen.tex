\subsection{Ebenenformen}
\label{sec:ebenen_formen}
\subsubsection{Koordinatenform}
\subsubsection{Normalenform}
\subsubsection{Parameterform}
Die Parameterform ist durch einen Stützvektor $\vec{p}$ und zwei Richtungsvektoren $\vec{h}$ und $\vec{k}$ definiert. \\
$
E: \vec{x} = \vec{p} + s\cdot\vec{h} + t\cdot\vec{k}
$
\\\\
Oder als Zahlenbeispiel (die $x_1 | x_2$ Ebene): \\
$
E: \vec{x} = 
\begin{Bmatrix}
    0 \\ 0 \\ 0
\end{Bmatrix}
 + s \cdot
\begin{Bmatrix}
    1 \\ 0 \\ 0
\end{Bmatrix}
 + t \cdot 
 \begin{Bmatrix}
    0 \\ 1 \\ ß
 \end{Bmatrix}
$
\\\\\\
Kann mithilfe des Kreuzprodukts zur Normalenform umgewandelt werdn: \\
\begin{itemize}
    \item $\vec{p}$: ist jeweils der Stützvektor
    \item $\vec{x}$: bleibt der Punkt
\end{itemize}


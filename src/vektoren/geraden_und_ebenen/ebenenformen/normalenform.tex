
\subsubsection{Normalenform}
\textbf{Erklärung}: 

Die Normalenform ist durch den Stützvektor $\vec{p}$, 
den Normalenvektor $\vec{n}$ und einen Vektor $\vec{x}$ definiert.
Dabei steht $\vec{x}$ für einen noch unbekannten Ortsvektor eines beliebigen Punktes auf der Ebene. 
In kurz: $\vec{x}$ ist ein beliebiger Punkt auf der Ebene. \\
Durch die Normalenform kann man die Koordinatenform leicht herleiten.

\textbf{Formel}: 
\[
\begin{pmatrix}
    (\vec{p} - \vec{x}) \bullet \vec{n_0}
\end{pmatrix}
\]
\textbf{Ausgeschrieben}:
\[
\begin{pmatrix}
    \begin{pmatrix}
        p_1 \\ p_2 \\ p_3
    \end{pmatrix} 
    - 
    \begin{pmatrix}
        x_1 \\ x_2 \\ x_3
    \end{pmatrix}
\end{pmatrix}
\bullet 
\begin{pmatrix}
    n_1 \\ n_2 \\ n_3
\end{pmatrix}
\]
Dabei steht: 
\begin{itemize}
    \item $\vec{p}$: Stützvektor
    \item $\vec{n}$: Normalenvektor
    \item $\vec{x}$: Punkt
\end{itemize}


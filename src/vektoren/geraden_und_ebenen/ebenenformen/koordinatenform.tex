
\subsubsection{Koordinatenform}
\underline{Erklärung}: \\
Durch die Koordinatenform lässt sich der Normalenvektor leicht ablesen. 
(in der Formel das $(a|b|c)$ ist der Normalenvektor).
Man kann auch einfach Testen ob ein Punkt auf der Ebene liegt indem man den Punkt in die $x$-Werte einsetzt und schaut ob die Gleichung stimmt.
\\\\
\underline{Formel}: \\
$
E: ax_1 + bx_2 + cx_3 = d
$
\\\\
\underline{Herleitung} (von Normalenform): \\
$
E: \vec{x} \bullet \vec{n} = \vec{p} \bullet \vec{n}
$
\\
Dabei steht: 
\begin{itemize}
    \item $\vec{x}$: einzugebender Punkt
    \item $\vec{p}$: Stützvektor
    \item $\vec{n}$: Normalenvektor
\end{itemize}

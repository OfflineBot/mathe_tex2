
\subsubsection{Parameterform}
\textbf{Erklärung}: 

Die Parameterform ist durch einen Stützvektor $\vec{p}$ und zwei Richtungsvektoren $\vec{h}$ und $\vec{k}$ definiert. 
Dabei legt der Stützvektor wie bei Geraden die Position im Raum (von $(0|0|0)$) fest. 
Der erste Richtungsvektor legt eine Gerade fest und der zweite legt die fest.
Mithilfe der Parameterform kann einfach die Normalenform hergeleitet werden.

\textbf{Formel}: 
\[
E: \vec{x} = \vec{p} + s\cdot\vec{h} + t\cdot\vec{k}
\]

\textbf{Ausgeschrieben}: 
\[
E: \vec{x} = 
\begin{pmatrix}
    p_1 \\ p_2 \\ p_3
\end{pmatrix}
 + s \cdot
\begin{pmatrix}
    h_1 \\ h_2 \\ h_3
\end{pmatrix}
 + t \cdot 
 \begin{pmatrix}
    k_1 \\ k_2 \\ k_3
 \end{pmatrix}
\]
Dabei steht:
\begin{itemize}
    \item $\vec{p}$: Stützvektor
    \item $\vec{h}$: Ein Richtungsvektor
    \item $\vec{k}$: Anderer Richtungsvektor (darf nicht identisch zu $\vec{h}$ sein)
\end{itemize}

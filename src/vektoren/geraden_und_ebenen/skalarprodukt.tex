\subsection{Skalarprodukt}
\label{sec:skalarprodukt}
Das Skalarpodukt ist eine Art der Vektormultiplikation.
Mit dieser Form lässt sich der 
\hyperref[sec:winkel_vektoren]{Winkel}
und somit auch die orthogonalität zweier Vektoren bestimmen. 
Allgemein kann man sagen, wenn das Skalarpodukt zweier Vektoren $0$ sind, sind diese orthogonal zueinander.
\par
Formel: \\
$
\vec{a} \bullet \vec{b} = |\vec{a}| \cdot |\vec{b}| \cdot cos(\alpha)
$
\par
Berechnung:
\begin{itemize}

    \item Allgemeine schreibweise: \\
    $
    \begin{Bmatrix}
        a_1 \\ a_2 \\ a_3
    \end{Bmatrix}
    \bullet 
    \begin{Bmatrix}
        b_1 \\ b_2 \\ b_3
    \end{Bmatrix}
    =
    a_1 \cdot b_1 + a_2 \cdot b_2 + a_3 \cdot b_3
    $
    \item Zahlenbeispiel: \\
    $
    \begin{Bmatrix}
        1 \\ 2 \\ 3
    \end{Bmatrix}
    \bullet
    \begin{Bmatrix}
    4 \\ 5 \\ 6
    \end{Bmatrix}
    = 
    1 \cdot 4 + 2 \cdot 5 + 3 \cdot 6 = 26
    $
\end{itemize}


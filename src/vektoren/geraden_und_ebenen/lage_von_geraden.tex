\subsection{Lage von Geraden}
\subsubsection{Lotfußpunkt Berechnen}
\underline{Bei windschiefen Geraden}: \\
Um den geringsten Abstand zweier Windschiefen geraden zu berechnen, 
erstellt man eine Hilfsebene aus einer Geraden indem man den zweiten Richtungsvektor für die Ebene aus dem Richtungsvektor aus der zweiten Geraden nimmt. 
Dann muss man einen beliebigen Abstand von Gerade und Ebene ausrechnen und hat damit das Ergebnis.
\par
\underline{Formel}: \\
Gegeben: \\
$
g: \vec{x} = \vec{p} + s \cdot \vec{u} \\
h: \vec{x} = \vec{q} + t \cdot \vec{v}
$
\\
Berechnung: \\
$
d(g; h) = |(\vec{q} - \vec{p}) \bullet \vec{n_0}|
$
\par
Dabei ist $\vec{n_0}$: $\frac{\vec{n}}{|\vec{n}|}$
\par
\secspacebig
\underline{Zahlenbeispiel}: \\
Gegeben: \\
$
g: \vec{x} = 
\begin{Bmatrix}
    9 \\ -8 \\ 6
\end{Bmatrix}
+ \lambda \cdot
\begin{Bmatrix}
    2 \\ -3 \\ 2
\end{Bmatrix} 
\\
h: \vec{x} = 
\begin{Bmatrix}
    0 \\ -1 \\ 1
\end{Bmatrix}
+ \mu \cdot 
\begin{Bmatrix}
    1 \\ -1 \\ 0
\end{Bmatrix}
$
\\
Berechnung: \\
$
d(g; h) = |(\vec{q} - \vec{p}) \bullet \vec{n_0}| \\
 = |
 \begin{Bmatrix}
    9 - 0 \\ -8 + 1 \\ 6 - 1
 \end{Bmatrix}
 = 
 \begin{Bmatrix}
    9 \\ -7 \\ 5
 \end{Bmatrix}
$
\\
Für $\vec{n}$ Kreuzprodukt anwenden von Richtungsvektoren (mehr dazu \hyperref[sec:kreuzprodukt]{hier}). \\
Ausgerechnet: \\
$
\vec{n} = 
\begin{Bmatrix}
    2 \\ 2 \\ 1
\end{Bmatrix}
\\
|\vec{n}| = 3
$
\\
Einsetzen: \\
$
\begin{Bmatrix}
    9 \\ -7 \\ 5
\end{Bmatrix}
\bullet
\begin{Bmatrix}
    2 \\ 2 \\ 1
\end{Bmatrix}
\cdot \frac{1}{3}
= 
3
$

\subsubsection{Schnittpunkt Berechnen}

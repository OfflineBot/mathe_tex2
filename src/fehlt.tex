\chapter{Das fehlt noch/muss ergänzt werden}
\section{BEISPIELE ÜBERALL}
\section{Allgemeines}
\begin{itemize}
    \item mittelwerte bei integralen
    \item analysis: funktionen und ihre graphen
    \item analysis: exponential und logarithmusfunktionen
    \item alles zu stochastik
    \item \textbf{Besonders Wichtig!}: Ebenenformen überarbeiten 
    \item logarithmusfunktionen
    \item Exponentialrechung - Parameter - Beispiele mit komplexen aufgaben
\end{itemize}

\section{Überschriften}
\begin{itemize}
    \item === Analysis ===
    \item 2.3.6 Parameter
    \item 2.4.2 Linearfaktordarstellung
    \item ff waagerechte und senkrechte Asymtoten
    \item ff Graph und Funktionsterm
    \item ff Untersuchen von Funktionsscharen
    \item ff Näherungsweise: Berechnen von Nullstellen
    \item 2.5.2 Lösungsmenge linearer Gleichungssysteme
    \item 2.5.3 Lineare Gleihcungssysteme mit Parametern auf der rechten Seite
    \item ff Besetimmen von ganz rantionaler Funktionen
    \item === Vektoren ===
    \item 3.1.7.2 Schnittpunkt Berechnen
    \item 3.1.8 Lage von Ebenen und Geraden
    \item 3.1.8.1 Durchstoßpunkt
    \item 3.1.9 Lage von Ebenen (Schnittgerade)
    \item 3.2.2 Abstand von Punkt zu Ebene
    \item 3.2.3 Spiegelung und Symmetrie
    \item 3.2.5 Schnittwinkel
    \item 3.2.7 Modellierung von geradlinigen Bewegungen
    \item 3.2.8 Vektorielle Beweise
    \item === Stochastik ===
    \item 4.1.1 Elementare Kombinatorik
    \item 4.1.2 Pfadregeln und Erwartungswerte
    \item 4.1.3 Bedingte Wahrscheinlichkeit
    \item ff Stochastische Unabhängigkeit
    \item ff Formel von Bernoulli und Binomialverteilung
    \item ff Erwartungswert und Histogramm
    \item ff Problemlösen mit der Binomalverteilung
    \item 4.2.1 Normalverteilung
    \item ff Gaußsche Glockenfunktion
    \item ff Sigma-Regeln
    \item ff Umkehraufgaben zur Normalverteilung
    \item ff Stetige Zufallsgrößen 
    \item 4.3.1 Einseitiger Hypothesentest
    \item ff Fehler beim Testen von Hypothesen
    \item ff Wahl der Nullhypothese
    \item Zweiseitiger Hypothesentest
\end{itemize}


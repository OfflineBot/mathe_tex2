\chapter{Das Fehlt/Klausurrelevant}

\section{Klausurrelevant}
\subsection{J2.1}
Basic:
\begin{multicols}{2}
    \begin{itemize}
        \item \hyperref[sec:ableiten]{Ableiten}
        \item \hyperref[sec:integrale]{Integrale} 
        \item \hyperref[sec:extremundwendepunkte]{Extrem- und Wendepunkte}
        \item \hyperref[sec:tangenteundnormale]{Tangente und Normale}
    \end{itemize}
\end{multicols}

Vektoren:
\begin{itemize}
    \item \hyperref[sec:hessscheform]{Hess'sche Normalenform}
\end{itemize}

Wahrscheinlichkeit (Neu):
\begin{itemize}
    \item \hyperref[sec:elemkombinatorik]{Basics}
    \item \hyperref[sec:bernoulli]{Bernoulli}
\end{itemize}

\subsection{J2.2}


\section{Fehlt}
\subsection{Allgemeines}
\begin{itemize}
    \item mittelwerte bei integralen
    \item analysis: exponential und logarithmusfunktionen
    \item \textbf{Besonders Wichtig!}: Ebenenformen überarbeiten 
    \item logarithmusfunktionen
    \item Exponentialrechung - Parameter - Beispiele mit komplexen aufgaben
\end{itemize}

\subsection{Überschriften}
\begin{itemize}
    \item === Analysis ===
    \item 2.3.6 Parameter
    \item 2.4.2 Linearfaktordarstellung
    \item ff waagerechte und senkrechte Asymtoten
    \item ff Graph und Funktionsterm
    \item ff Untersuchen von Funktionsscharen
    \item ff Näherungsweise: Berechnen von Nullstellen
    \item 2.5.2 Lösungsmenge linearer Gleichungssysteme
    \item 2.5.3 Lineare Gleihcungssysteme mit Parametern auf der rechten Seite
    \item ff Besetimmen von ganz rantionaler Funktionen
    \item === Vektoren ===
    \item 3.1.8.1 Durchstoßpunkt
    \item 3.1.9 Lage von Ebenen (Schnittgerade)
    \item 3.2.3 Spiegelung und Symmetrie
    \item 3.2.5 Schnittwinkel
    \item 3.2.7 Modellierung von geradlinigen Bewegungen
    \item 3.2.8 Vektorielle Beweise
    \item === Stochastik === (neu)
\end{itemize}

